\documentclass[a4paper,twoside]{article}


\usepackage{apalike}
\usepackage{SCITEPRESS}

\usepackage{multicol}
\usepackage[small]{caption}


\begin{document}


%%%%%%%%%%%%%%%%%%%%%%%%%%%%%%%%%%%%%%%%%%%%%%%%%%%%%%%%%%%%%%%%%%%%%%%%%%%%%%%%
%%%                               80 COLONNES                                %%%
%%%%%%%%%%%%%%%%%%%%%%%%%%%%%%%%%%%%%%%%%%%%%%%%%%%%%%%%%%%%%%%%%%%%%%%%%%%%%%%%


\title{RIOT OS Paves the Way for Implementation of High-Performance
       MAC Protocols}


%\author{
%\authorname{K\'evin Roussel, Ye-Qiong Song and Olivier Zendra}
%\affiliation{LORIA/INRIA Nancy Grand-Est,\\
%             Universit\'e de Lorraine,\\
%             615, rue du Jardin Botanique,\\
%             54600 Villers-L\`es-Nancy, France}
%\email{\{Kevin.Roussel,Ye-Qiong.Song,Olivier.Zendra\}@inria.fr}
%}


\keywords{Hard Real-Time, Internet of Things, MAC protocols,
          Embedded Operating Systems, RIOT OS}


\abstract{Implementing new, high-performance MAC protocols requires
real-time features, to be able to synchronize correctly between different
unrelated devices. Such features are highly desirable for operating
wireless networks that are designed to be part of the Internet of Things
(IoT). Unfortunately, the existing operating systems used in this domain
cannot provide such features. On the other hand, ``bare-metal'' development
sacrifices portability, as well as the multitasking abilities needed
to develop the rich applications that are useful in the domain of
the Internet of Things.\\
We describe in this paper the development of a port of RIOT OS
on the MSP430 microcontroller architecture widely used in IoT-enabled motes.
RIOT OS offers rich and advanced  real-time features, especially the
simultaneous use of as many hardware timers as the underlying platform (MCU)
can offer. We then demonstrate the power of these features by providing
a new implementation of S-CoSenS, an efficient MAC protocol that uses
very low processing power and energy.}


\onecolumn \maketitle \normalsize \vfill


\section{\uppercase{Introduction}}

To implement new, high-performance MAC protocols, one needs to be able to react
to events with good reactivity (lowest latency possible) and flexibility.

This implies to have efficient systems to manage interruptions, and the extensive
use of hardware timers.

\section{\uppercase{Problem statement}}

Problem: current OSes, and especially the reference, Contiki \cite{Contiki},
is unable to provide the needed features: Contiki is no real-time OS,
the processing of ``events'' (using Contiki's terminology) uses a slow
cooperative scheduler, with too much latency (8 ms $= 8000 \mu$s).
The only real-time feature, \texttt{rtimer}, is much too limited: only one
instance, and Contiki's kernel functions must not be called from this
mechanism.
The other solution, ``bare-metal'' programming, is also very unsatisfactory:
no portability, no multitasking, and the need to redevelop every needed tool
and feature,  making application programming harder and tedious.

\section{\uppercase{RIOT OS: the new solution}}

\cite{RIOT}
All the benefits of an OS, plus...
Provides all the features that other OSes miss:
pre-emptive multitasking, priority aware task scheduler;
efficient tickless kernel, with low-latency interrupt management;
highly efficient use of hardware timers: all of them are used concurrently,
offering the ability to schedule actions with high granularity (on MSP430,
32 microseconds of precision!).
Bonus: uses standard C, no limitations (e.g. like proto-threads forbidding
the use of \texttt{select});
Very modular design, uses little memory space.

\section{\uppercase{Our contribution}}

MSP430 is a microcontroller (MCU) architecture, offering very low-power
consumption, cheap price, and good performance thanks to a custom 16-bit
RISC design. This architecture is very common in IoT motes.
It is also very well supported, especially by the Cooja simulator
provided by the Contiki project \cite{Contiki}.

Make the MSP430 port really robust and usable, by:
porting to an industrial device (Zolertia Z1), and
correcting the hardware-dependent part of the scheduler on MSP430.

\section{\uppercase{Use Case: implementing the S-CoSenS MAC protocol}}

Excellent synchronization, thanks to HW timers use.

Very sparse use of both transceiver AND processing power,
should spare much energy.

\section{\uppercase{Conclusion}}

A discuter.

\vfill
\bibliographystyle{apalike}
{\small
\bibliography{sensornets2015}}


\end{document}
